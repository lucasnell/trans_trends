\documentclass[12pt]{article}
% \usepackage[sc]{mathpazo} % Like Palatino with extensive math support
\usepackage{mathptmx} % Like Times New Roman
\usepackage{fullpage}
\usepackage[authoryear,sectionbib,sort]{natbib}
\linespread{2}
\usepackage[utf8]{inputenc}
\usepackage{lineno}
\usepackage{titlesec}
\titleformat{\section}[block]{\Large\bfseries\filcenter}{\thesection}{1em}{}
\titleformat{\subsection}[block]{\Large\itshape\filcenter}{\thesubsection}{1em}{}
\titleformat{\subsubsection}[block]{\large\itshape}{\thesubsubsection}{1em}{}
\titleformat{\paragraph}[runin]{\itshape}{\theparagraph}{1em}{}[. ]\renewcommand{\refname}{Literature Cited}

% For Icelandic ð symbol:
\DeclareTextSymbolDefault{\dh}{T1}
% Increased spacing in math mode:
\medmuskip=8mu % by default it is equal to 4 mu
\thickmuskip=10mu % by default it is equal to 5 mu

% Figures
\usepackage{graphicx}
% Table
\usepackage{booktabs}


%%%%%%%%%%%%%%%%%%%%%
% Line numbering
%%%%%%%%%%%%%%%%%%%%%
%
% Please use line numbering with your initial submission and
% subsequent revisions. After acceptance, please turn line numbering
% off by adding percent signs to the lines %\usepackage{lineno} and
% to %\linenumbers{} and %\modulolinenumbers[3] below.
%
% To avoid line numbering being thrown off around math environments,
% the math environments have to be wrapped using
% \begin{linenomath*} and \end{linenomath*}
%
% (Thanks to Vlastimil Krivan for pointing this out to us!)

\title{Linking taxon and community responses to environmental
predictors in time and space}

% This version of the LaTeX template was last updated on
% November 8, 2019.

%%%%%%%%%%%%%%%%%%%%%
% Authorship
%%%%%%%%%%%%%%%%%%%%%
% Please remove authorship information while your paper is under review,
% unless you wish to waive your anonymity under double-blind review. You
% will need to add this information back in to your final files after
% acceptance.

\author{
Joseph S. Phillips$^{1,2,3,\dagger}$ \\
Lucas A. Nell$^{1,4,\dagger}$ \\
Jamieson C. Botsch$^{1}$}



\usepackage{amsmath} % for split math environment


\date{}

\begin{document}

\raggedright
\setlength\parindent{0.25in}

\maketitle


\noindent{} 1. Department of Integrative Biology, University of Wisconsin, Madison, Wisconsin 53706 USA

\noindent{} 2. Department of Aquaculture and Fish Biology, H\'{o}lar University, Skagafj\"{o}r{\dh}ur 551 Iceland

\noindent{} 3. E-mail: josephsmphillips@gmail.com

\noindent{} 4. E-mail: lucas@lucasnell.com

\noindent{} $\dagger$ Both authors contributed equally.



\bigskip

% \textit{Manuscript elements}: %Figure~1, figure~2, table~1, online appendices~A and B (including $-- figure~A1 and figure~A2). Figure~2 is to print in color.

\bigskip


\textit{Keywords}: {allochthonous subsidies; autoregressive model; linear mixed model;
M\'{y}vatn; time-series analysis; tundra arthropods
}


\bigskip

\textit{Manuscript type}: Statistical Report.

% \bigskip
% \noindent{\footnotesize Prepared using the suggested \LaTeX{} template for \textit{Am.\ Nat.}}

\linenumbers{}
% \modulolinenumbers[3]

\newpage{}

% ---------------------------------------------------------------------------------------
% ---------------------------------------------------------------------------------------
% Abstract
% ---------------------------------------------------------------------------------------
% ---------------------------------------------------------------------------------------






\newpage{}



% ---------------------------------------------------------------------------------------
% ---------------------------------------------------------------------------------------
% Introduction
% ---------------------------------------------------------------------------------------
% ---------------------------------------------------------------------------------------



\section*{Introduction}








% ---------------------------------------------------------------------------------------
% ---------------------------------------------------------------------------------------
% Methods
% ---------------------------------------------------------------------------------------
% ---------------------------------------------------------------------------------------




\section*{Methods}








% ---------------------------------------------------------------------------------------
% ---------------------------------------------------------------------------------------
% Results
% ---------------------------------------------------------------------------------------
% ---------------------------------------------------------------------------------------



\section*{Results}

There was a positive response of predator abundance to midge deposition,
as indicated by both the coefficient estimates (Figure \ref{fig:coefs}A,C)
and the modestly large improvement of the full model
relative to the reduced model (LOO deviance in Table \ref{tab:model-summary}).
The response to midges was somewhat variable across taxa,
as judged by the taxon-specific coefficients (Figure \ref{fig:coefs}A)
and the posterior distribution
for the random effect standard deviation (Figure \ref{fig:coefs}D).
However, this variation among taxa made a relatively small contribution
to the model fit (Table \ref{tab:model-summary}).
In contrast to the response to midges, there was large variation among taxa in
linear trends through time and distance from the lake (Figure \ref{fig:coefs}A,D),
and this taxon-specific variation dominated the contribution to the model fit
(Table \ref{tab:model-summary}).
Overall, the magnitudes of the taxon-specific responses to time and distance
were larger than the respones to midges.



The AR coefficients were generally similar across taxa and of modest magnitude,
indicating moderate levels of temporal autocorrelation across years
(Figure \ref{fig:coefs}B).
Ground spiders were an exception,
with an AR coefficient near zero indicating limited autocorrelation.
The relatively low autocorrelation is apparent from the time series,
where ground spiders appear more tightly constrained to their mean abundance
than was the case for other taxa (Figure \ref{fig:obs-data}A),
which results in less autocorrelated (i.e. ``faster'') fluctuations through time
\citep{Ziebarth2010}.



The first three PC axes contained all of the variation
due to linear effects of the three predictors and
collectively accounted for $\sim$68\% of the observed community variation
in abundance (Table \ref{tab:model-summary}).
Distance made the largest overall contribution,
followed by time and midges.
The effect of distance manifested almost exclusively through PC1,
while the effects of midges and time were both split between PC2 and PC3.
This indicates some degree of correspondence between linear responses of the community
to midges and time, as can be seen in taxon-specific coefficients
(Figure \ref{fig:coefs}A).
In contrast, the community response to distance
was largely uncorrelated with the responses to time and midges.



Although community-level results can be inferred from Figure \ref{fig:obs-data} and
Table \ref{tab:model-summary}, Figure \ref{fig:pca} illustrates how predictor effects
on the community can be visualized in a way similar to conventional ordination or
dimension-reduction methods widely used in community ecology.

The taxon-response vectors (Figure \ref{fig:pca}A)
provided information on the variation in overall abundance versus composition per se;
vectors of similar direction and magnitude
imply consistent responses across taxa (i.e. overall abundance),
while vectors of either different direction or magnitude imply variation in composition.
The direction of the taxon vectors was largely similar along PC2 (all positive
except for sheet weavers), reflecting the consistent overall
response of the community to midge deposition (Figure \ref{fig:pca}D).
In contrast, the direction and magnitude
of the taxon-vectors along PC1 was quite heterogeneous,
reflecting the fact that there was large variation in composition with
distance from the lake (Figure \ref{fig:pca}C).
Time had a similar effect on community composition per se, as can be seen when
projecting data onto PC2 and PC3 (Figure S1).
This contrast between effects on overall abundance versus composition
resulted directly from the relative magnitudes
of fixed versus taxon-specific random effects (Figure \ref{fig:coefs}),
illustrating the link between the two organizational scales.






% ---------------------------------------------------------------------------------------
% ---------------------------------------------------------------------------------------
% Discussion
% ---------------------------------------------------------------------------------------
% ---------------------------------------------------------------------------------------




\section*{Discussion}

In this study, we show how linear mixed models with temporal autocorrelation can be used
to quantify community responses to environmental variation
from replicated time series observations.
Our approach is based on previous studies that used mixed models
for estimating taxon- and community-level responses to environmental variation
through space \citep{Jackson2012, Bartrons2015}.
We extend this approach for community time series by incoporating temporal autocorrelation
with group-specific values of the autoregressive parameter,
which is important as populations with different dynamics
likely have different degrees of autocorrelation.
Furthermore, we show how predicted values from the model estimates can be combined with
with principle components analysis \citep[following][]{Jackson2012} to visualize
community composition along axes of variation most strongly associated with environmental
variation.
This analysis makes explicit the connections betweeon taxon- and community-level variation
and provides a conceptual link between the mixed model approach
and conventional ordination methods.


To illustrate its utility,
we used autoregressive mixed models to quantify the response
of a predatory arthropod community to spatiotemporal variation in allochthonous resources
at Lake M\'{y}vatn in northern Iceland.
We found a positive overall response of predator activity-density to midge deposition,
which is consistent with previous studies at M\'{y}vatn
\citep{Hoekman2011, Dreyer2012, Sanchez2018, Hoekman2019}
and expected from studies of allochthonous subsidies in other systems
\citep{}.
However, the variation among taxa in their reponses to midges was fairly limited,
which means that variation in midge deposition was primarily associated with
changes in overall abundance in the community, rather than composition per se.
In contrast, there was large variation in taxon-specific trends in activity-density
through time and space,
which manifested as spatiotemporal variation in community composition.

We found that most of the predatory arthropods showed modest levels of temporal
autocorrelation in activity density,
except for ground spiders which showed very low autocorrelation.
While we caution again strong conclusions regarding dynamic processes based on short
time series,
these results suggest that ground spiders returned more rapidly to their mean
acitivyt-density,
which in turn manifested "faster" flucutations through time \citep{}.
This could reflect ecological differences between grounds spiders and the other taxa,
for example in life history or in movement behavior.
Temporal autocorrelation can also be influence by species interactions,
which are not included in our model formulation.
When long time series are available,
explicit models of community dynamics can be used to infer species interactions
and so are more approach than the approach presented here.
Nonetheless, when the primary goal is to infer general responses environmental variation
(as is often the case),
our approach provides some additional insight into dynamic processes
when interpreted with caution.

[Final short paragraph to wrap things up?]
















% ---------------------------------------------------------------------------------------
% ---------------------------------------------------------------------------------------
% Acknowledgments
% ---------------------------------------------------------------------------------------
% ---------------------------------------------------------------------------------------
% You may wish to remove the Acknowledgments section while your paper
% is under review (unless you wish to waive your anonymity under
% double-blind review) if the Acknowledgments reveal your identity.
% If you remove this section, you will need to add it back in to your
% final files after acceptance.

% \section*{Acknowledgments}
%
% OEC would like to thank the world. GHC is much indebted to the solar system. AQE was supported by a generous grant from the Milky Way (MW/01010/987654).


% ---------------------------------------------------------------------------------------
% ---------------------------------------------------------------------------------------
% Appendices
% ---------------------------------------------------------------------------------------
% ---------------------------------------------------------------------------------------

% \newpage{}
%
% \input{app_A}


% ---------------------------------------------------------------------------------------
% ---------------------------------------------------------------------------------------
% Literature Cited
% ---------------------------------------------------------------------------------------
% ---------------------------------------------------------------------------------------



\bibliographystyle{ecology.bst}
\newpage{}
\bibliography{refs.bib}


\clearpage

\begin{table}
\caption{\label{fig:obs-data}Model summary.}
\begin{tabular}{rrrrrrrr}
\toprule
 & \multicolumn{2}{c}{LOO deviance} & & \multicolumn{4}{c}{Variance partitioning} \\
 \cmidrule{2-3} \cmidrule{5-8}
 & \multicolumn{1}{c}{Taxon-variation} & \multicolumn{1}{c}{Overall} & &
    \multicolumn{1}{c}{PC1} & \multicolumn{1}{c}{PC2} & \multicolumn{1}{c}{PC3} &
    \multicolumn{1}{c}{Total} \\
& \multicolumn{1}{c}{(random)} & \multicolumn{1}{c}{(fixed + random)} & &
    \multicolumn{1}{c}{(0.34)} & \multicolumn{1}{c}{(0.17)} &
    \multicolumn{1}{c}{(0.17)} & \multicolumn{1}{c}{(0.68)} \\
\midrule
time & 37 & 34 &  & 0.00 & 0.28 & 0.84 & 0.19\\
distance & 64 & 64 &  & 0.99 & 0.04 & 0.01 & 0.34\\
midges & 4 & 15 &  & 0.01 & 0.67 & 0.15 & 0.14\\
\bottomrule
\end{tabular}
\end{table}


\clearpage

\begin{figure}
\centering
\includegraphics{../analysis/output/fig1.pdf}
\caption{\label{fig:obs-data}
Community time series data for 6 predatory arthropod taxa at Lake M\'{y}vatn
(A) Standardized activity-density of arthropods through time.
Narrow gray lines are time series grouped by site and distance.
(B) Standardized activity-density of arthropods across distance from the lake.
Thick red lines are midge catch averaged by (A) year or (B) distance.
Thick blue lines are standardized activity-density averaged by taxon and
(A) year or (B) distance.
}
\end{figure}




\clearpage

\begin{figure}
\centering
\includegraphics{../analysis/output/fig2.pdf}
\caption{\label{fig:coefs}
Coefficient figure.
}
\end{figure}



\clearpage

\begin{figure}
\centering
\includegraphics{../analysis/output/fig3.pdf}
\caption{\label{fig:pca}
PCA figure.
(A) PCs multiplied by 2.
(B--D) PCs multiplied by 5.
}
\end{figure}



\end{document}
