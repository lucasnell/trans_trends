\documentclass[12pt]{article}
% \usepackage[sc]{mathpazo} % Like Palatino with extensive math support
\usepackage[letterpaper, margin=1in]{geometry}
% \usepackage{mathptmx} % Like Times New Roman
\usepackage{newtxtext,newtxmath}
\usepackage{fullpage}
\usepackage[authoryear,sectionbib,sort]{natbib}

% The following works for line spacing for ESA journals
% (per http://esapubs.org/esapubs/latexTIPsESA.pdf):
\linespread{1.9}

\usepackage[utf8]{inputenc}
\usepackage{lineno}
\usepackage{titlesec}
\titleformat{\section}[block]{\Large\bfseries\filcenter}{\thesection}{1em}{}
\titleformat{\subsection}[block]{\Large\itshape\filcenter}{\thesubsection}{1em}{}
\titleformat{\subsubsection}[block]{\large\itshape}{\thesubsubsection}{1em}{}
\titleformat{\paragraph}[runin]{\itshape}{\theparagraph}{1em}{}[. ]\renewcommand{\refname}{Literature Cited}

% For Icelandic ð symbol:
\DeclareTextSymbolDefault{\dh}{T1}
% Increased spacing in math mode:
\medmuskip=8mu % by default it is equal to 4 mu
\thickmuskip=10mu % by default it is equal to 5 mu

% Figures
\usepackage{graphicx}
% Table
\usepackage{booktabs}


%%%%%%%%%%%%%%%%%%%%%
% Line numbering
%%%%%%%%%%%%%%%%%%%%%
%
% Please use line numbering with your initial submission and
% subsequent revisions. After acceptance, please turn line numbering
% off by adding percent signs to the lines %\usepackage{lineno} and
% to %\linenumbers{} and %\modulolinenumbers[3] below.
%
% To avoid line numbering being thrown off around math environments,
% the math environments have to be wrapped using
% \begin{linenomath*} and \end{linenomath*}
%
% (Thanks to Vlastimil Krivan for pointing this out to us!)

\title{Quantifying community responses to environmental variation from replicate
time series}



% This version of the LaTeX template was last updated on
% November 8, 2019.

%%%%%%%%%%%%%%%%%%%%%
% Authorship
%%%%%%%%%%%%%%%%%%%%%
% Please remove authorship information while your paper is under review,
% unless you wish to waive your anonymity under double-blind review. You
% will need to add this information back in to your final files after
% acceptance.

\author{
Joseph S. Phillips$^{1,2,3,\dagger}$ \\
Lucas A. Nell$^{1,4,\dagger}$ \\
Jamieson C. Botsch$^{1}$}



\usepackage{amsmath} % for split math environment


\date{}

% so that no commas are used in citations:
\bibpunct{(}{)}{,}{a}{}{,}


\begin{document}

\raggedright
\setlength\parindent{0.25in}

\maketitle


\noindent{} 1. Department of Integrative Biology, University of Wisconsin, Madison, Wisconsin 53706 USA

\noindent{} 2. Department of Aquaculture and Fish Biology, H\'{o}lar University, Skagafj\"{o}r{\dh}ur 551 Iceland

\noindent{} 3. E-mail: joseph@holar.is

\noindent{} 4. E-mail: lucas@lucasnell.com

\noindent{} $\dagger$ Both authors contributed equally.



\bigskip

Running head: {Community responses from time series}

% \textit{Manuscript elements}: %Figure~1, figure~2, table~1, online appendices~A and B (including $-- figure~A1 and figure~A2). Figure~2 is to print in color.

\linenumbers{}
% \modulolinenumbers[3]

\clearpage

% ---------------------------------------------------------------------------------------
% ---------------------------------------------------------------------------------------
% Abstract
% ---------------------------------------------------------------------------------------
% ---------------------------------------------------------------------------------------


\section*{Abstract}

Times-series data for ecological communities are increasingly available
from medium- and long-term studies designed to track responses
to environmental change. However, classical multivariate methods for
analyzing community composition are generally inappropriate for time series,
as they do not account for temporal autocorrelation
in the abundance of members of the community.
Furthermore, these traditional approaches often obscure the connections between
taxon- and community-level responses, limiting the capacity to
infer the mechanisms of community change.
We show how linear mixed models with group-specific temporal autocorrelation
can be be used to infer taxon- and community-level responses
to predictor variables from replicated time series data.
Taxon-specific responses to environmental predictors are modeled
using random effects, and these responses can be used to characterize
variation in community abundance and composition.
Furthermore, the degree of autocorrelation is
estimated separately for each taxon
to account for potential differences in their dynamics.
We illustrate the utility of the approach
by analyzing the response of a predatory arthropod community to
spatiotemporal variation in allochthonous resources in a tundra landscape.
Our results show how mixed models with temporal autocorrelation provide
a unified approach to characterizing taxon and community-level
responses to environmental variation through time and space.


\bigskip

\textit{Keywords}: {allochthonous subsidies; autoregressive model; linear mixed model;
M\'{y}vatn; time-series analysis; tundra arthropods
}





% ---------------------------------------------------------------------------------------
% ---------------------------------------------------------------------------------------
% Introduction
% ---------------------------------------------------------------------------------------
% ---------------------------------------------------------------------------------------



\section*{Introduction}


A central goal of ecology is to understand how communities vary through time, and
how this variability is shaped by the environment.
Time series data are increasingly being used to assess temporal trends in communities.
Because environmental variables also change through time, time series data provide
additionally valuable information for how communities change in response to the
environment. For example...

The response of ecological communities to environmental predictors is often analyzed using
ordination methods (e.g.), which map variation in abundance or occurnace onto orthogonal
axes that provide synoptic assessments of community variation. However, these methods
are generally inapporpriate for time series data, as they do not appropriately account
for temporal autocorrelation in the abundance of members of the community.
While some time-series methods have been developed for analyzing ecological communities,
these have generally focused on inferring interactions between species and therefore
require very long time series. In contrast, community time series are often short but
contain replication through space or across experimental units, with the goal of inferring
the responses of communities to external drivers or experimental manipulations.

Here, we show linear mixed models with temporal autocorrelation structures
(hereafter "autoregressive mixed models") can be be used to infer taxon- and
community-level responses to predictor variables from time series data. The
approach is based on the method of \cite{Jackson2012}, which models variation in
taxon-specific responses to predictors as random effects. These taxon-specific responses
can then be translated to the community level, thereby providing a unified statistical
framework for understanding the ecological responses to environmental drivers at multiple
organizational scales. We extend this approach by showing how linear mixed models can be
formulated with separate temporal autocorrelation structures for different groupings
(i.e. taxa), accounting for the fact that different taxa are likely to be characterized by
different dynamics.

We illustrate the utility of this approach by analyzing the responses of predatory
arthropods to spatiotemporal variation in allothchonous resources
at Lake M\`{y}vatn in northern Iceland. M\'{y}vatn has large emergences of midges
(Diptera: Chironomidae) that serve as food for the terrestrial arthropods.
The midges have large interannual fluctuations and decline in deposition
with distance from the lakeshore, and we were interested in separating the response
of the community to midge deposition from variation in time and space.
This study shows how autoregressive mixed models can be used to dissentangle
taxon- and community-level responses to multiple predictors from time-series data.











% ---------------------------------------------------------------------------------------
% ---------------------------------------------------------------------------------------
% Methods
% ---------------------------------------------------------------------------------------
% ---------------------------------------------------------------------------------------




\section*{Methods}

We define a vector $\mathbf{Y}$, consiting of the standardized abundances for each taxon in each plot through time.
The rows are grouped as individual time series (i.e. taxon-plot combinations), with observations within each group
ordered according to time. From this vector, we define two additional vectors: $\mathbf{y}$ omits the first value from each
time-series, while $\mathbf{y\sp{\prime}}$ omits the final value and therefore is lagged by one time step (for each time series)
relative $\mathbf{y}$. This allows us to specify a continuous-time lag-1 autoregressive model for the $i$th
observation of $\mathbf{y}$ as

\begin{equation} \label{eq:y-i}
\begin{gathered}
y_i~=~\mathbf{x}_i~{\boldsymbol\beta}_i~+~\phi_{\text{taxon}[i]}^{\text{time}_i-\text{time}\sp{\prime}_i}~y\sp{\prime}_i
        +~\epsilon_i \\
\epsilon_i~\sim~\mathcal{N}\left(0,~\sigma_{\text{residual}}\right),
\end{gathered}
\end{equation}

\noindent where $\mathbf{x}_i$ is a row vector of predictor values, ${\boldsymbol\beta}_i$ is a column vector of
regression coefficients, $\phi_{\text{taxon}[i]$ is the taxon-specific autoregressive (AR) parameter, $\text{time}_i$ is
the current time, $\text{time}\sp{\prime}_i$ is the lagged time (defined as for  $\mathbf{y\sp{\prime}}$), and $\epsilon_i$
is the Gaussian residual error with standard deviation $\sigma_{\text{residual}}$. Because the abundances $y_i$ are log-transformed,
equation \ref{eq:y-i} is essentially log-linear model of population dynamics, with the intrinsic growth rate given by
$\mathbf{x}_i{\boldsymbol\beta}_i$. The primary difference between equation \ref{eq:y-i} and a more traditional AR model is the use
of separate values of the AR paramter for different groups (in this case, different taxa). This is an important extension, as it
allows different members of a community with different dynamics (and likely different levels of autocorrelation) to be included in a
single model. We parameterized the model with the predictors giving the change in $y_i$, rather than the mean of the stationary
distribution as is commonly done, because this helped to reduce the correlation between the estimates of the regression coefficients
and AR parameters.

We were interested in characterizing the response of the predatory arthropod community to midge depositon, while also
accounting for potential trends through time and space due to other factors. Therefore, we defined the vector of predictor
values for the $i$th observation of $\mathbf{y}$ as

\begin{equation} \label{eq:x-vec}
\mathbf{x}_i~=~\begin{bmatrix} 1 & \text{midge}_i & \text{time}_i & \text{dist}_i \end{bmatrix},
\end{equation}

\noindent with the vector of corresponding coefficients defined as

\begin{equation} \label{eq:beta-vec}
{\boldsymbol\beta}_i~=~\begin{bmatrix} {}^1\hspace*{-1pt}\beta_i \\ {}^\text{midge}\hspace*{-1pt}\beta_i \\
    {}^\text{time}\hspace*{-1pt}\beta_i \\ {}^\text{dist}\hspace*{-1pt}\beta_i \end{bmatrix}.
\end{equation}

We modeled the coefficients hierarchically as

\begin{equation} \label{eq:betas}
\begin{gathered}
{}^1\hspace*{-1pt}\hspace*{-1pt}\beta_i~=~{}^1\hspace*{-1pt}\alpha~+~{}^1\hspace*{-1pt}\zeta_{\text{taxon}[i]}~+~
        {}^1\hspace*{-1pt}\zeta_{\text{trans}[i]}~+~{}^1\hspace*{-1pt}\zeta_{\text{plot}[i]} \\
{}^\text{midge}\hspace*{-1pt}\beta_i~=~{}^\text{midge}\hspace*{-1pt}\alpha~+~
        {}^\text{midge}\hspace*{-1pt}\zeta_{\text{taxon}[i]} \\
{}^\text{time}\hspace*{-1pt}\beta_i~=~{}^\text{time}\hspace*{-1pt}\alpha~+~
        {}^\text{time}\hspace*{-1pt}\zeta_{\text{taxon}[i]} \\
{}^\text{dist}\hspace*{-1pt}\beta_i~=~{}^\text{dist}\hspace*{-1pt}\alpha~+~
        {}^\text{dist}\hspace*{-1pt}\zeta_{\text{taxon}[i]}
\end{gathered}
\end{equation}

\noindent with fixed effects ${}^k\hspace*{-1pt}\alpha$ and random effects ${}^k\hspace*{-1pt}\zeta_{\text{g}[j]}$ for predictor $k$ and the $j$th level of grouping variable $g$. The random effects where modeled with Gaussian distributions with standard error $\sigma_g$:

\begin{equation} \label{eq:zetas}
{}^k\hspace*{-1pt}\zeta_{g[j]}~\sim~\mathcal{N}\left(0,~{}^k\hspace*{-1pt}\sigma_g\right)
\end{equation}

Using conventional mixed model terminology, this model includes random intercepts grouped by taxon, transect (notated
"trans"), and plot, and random slopes for midges, time, and distance grouped by taxon. The fixed slopes give the overall
response across taxa to the predictor variables, while the corresponding random effects give the deviation for each taxon from
this overall response. The overall response of each taxon is the sum of the fixed and random components.
Characterizing the responses for each taxon in a single model using random effects allows for "partial pooling" of the
taxon-specific responses towards the mean response, which reduces the noisiness of individual estimates and ameliorates issues
related to multiple comparisons when comparing the taxon specific responses (either to each other or to 0).

Variation in community composition per se arises from variation in the taxon-specific responses, and the magnitude of this
variation for predictor $k$ is given by ${}^k\hspace*{-1pt}\sigma_\text{taxon}$. Therefore, the model characterizes both
taxon-level and community-level responses to the predictors. To make the community responses more explicit, we
generated predicted values (assuming no autocorrelation) $\hat{\mathbf{y}}$ indexed by $r$ as

\begin{equation} \label{eq:preds}
\hat{y}_r~=~\hat{\mathbf{x}}_r~\hat{{\boldsymbol\beta}}_r
\end{equation}

\noindent where $\hat{\mathbf{x}}_r$ is the $r$th row from a data frame defined analagously to equation \ref{eq:x-vec} but
with values evenly spaced across the observed range of predictors, and $\hat{{\boldsymbol\beta}}_r$ is a
corresponding vector of coefficients using the estimated fixed and random effects as in equation \ref{eq:betas}
but excluding the random intercepts grouped by transect and plot. We then converted these predicted values into a
"replicate-by-taxon" matrix (analagous to a "site-by-species" matrix) and performed a Principle Components Analysis (PCA)
to generate orthogonal axes characterizating variation in the community. Finally, we projected the observed data onto these
axes. This allowed us to visualize variation in the community, both in terms of composition per se and overall abundance,
along axes of variation most associated with the predictor variables of the model (i.e. midges, time, and distance). This
is analagous to more conventional ordination approaches (e.g. RDA), but makes explicit the connection to taxon-specific
responses. Furthermore, it takes advantage  of the statisitcal machinery of regression-type models, which allows for the
formal accounting of temporal autocorrelation (i.e. equation \ref{eq:y-i}) that is import for valid statistical inference
when using time-series data.

While the focus of our analysis was on the inferences provided by the full model specified by equations \ref{eq:y-i}-\ref{eq:zetas},
we also compared the fit of the full model to reduced models excluding responses to the predictor variables. We excluded predictors
in two ways: (1) excluding only the taxon-specific random effect or (2) excluding both the fixed and random effect. The comparison
of the full model to the former provided inference of the effect of the predictor on community composition per se, while comparison
to the latter provided inference on the effect of the predictor overall. Becuase the goal was to provide inference on the full
model, rather than identify an "optimal" model, we only excluded single predictor variables. We compared the models using the
Leave-On-Out information criterion, which is analagous to other information criteria such as the Akaike and Bayesian infromation
criteria (AIC and BIC, respectively) and is reported in units of deviance and so has a simialr interpretation. However, it utilizes
information across the full posterior and therefore is fully Bayesian.

We fit the models in a Bayesian framework using Stan, although this method could be readily adapted to other model fitting software increasingly used by ecologists such as WinBUGS, JAGS, and AD Model Builder. We used Gaussian distributions with mean 0 and standard deviation 1 as weakly-informative priors for fixed effects, random effects, and standard deviations, with the latter truncated as 0 as standard deviations must be positive. We used a similar prior for the AR parameter, but with standard deviation 0.5 to aid with convergence towards stationarity (i.e. $|\phi_{\text{taxon}[i]}|<1$). We truncated the lower bound of the prior at 0 to prevent negative values that would give quasi-cyclic behviors, but we did not provide a hard constraint to the upper bound. We assessed convergence using the standard diagnostics provided by Stan, including effective sample size of the Markov chain, number of divergence transitions, and potential scale reduction factor; all of these fell within acceptable bounds as defined by Stan's defaults.






% ---------------------------------------------------------------------------------------
% ---------------------------------------------------------------------------------------
% Results
% ---------------------------------------------------------------------------------------
% ---------------------------------------------------------------------------------------



\section*{Results}

There was a positive overall response of predator activity-density to midge deposition,
as indicated by both the estimated mean response
(Figure \ref{fig:coefs}A,C)
and the modestly large LOO deviance (Table \ref{tab:model-summary}).
The variation among taxa in the reponses to midges was meaningfuly different from zero
(Figure \ref{fig:coefs}D)
but contributed relatively little to the
model fit (Table \ref{tab:model-summary}).
In contrast to the response to midges,
there was large variation among taxa in trends through time and space,
and this taxon-variation dominated the contribution to the model fit.
Some taxa with strongly positive responses to midges (e.g. ground spiders)
increased with distance from the lake,
despite the fact that midge deposition declined with distance.
This underscores the value of including multiple predictors in a single model
so that their respective effects can be disentangled.

The AR coefficients were generally simliar across taxa and of modest magnitude,
indicating moderate levels of temporal autocorrelation across years.
Ground spiders were an exception,
with an AR coefficient bording zero indicating limited autocorrelation.
The relatively low autocorrleation is apparent from the time series,
where ground spiders appear more tightly constrianed to their mean abundance
than was the case for other taxa (Figure \ref{fig:obs-data}).
Note that while the low AR coefficent for ground spiders was associated
with by far the most positive trend with time,
the \textit{magnitude} of this trend was not much larger than inferred for some other taxa,
suggesting that it was not a statistical artifact of limited identifiability.

The first three PC axes contained all of the variation
due to linear effects of the three predictors and
collectively accounted for over half of the observed community variation
in activity-density (Table \ref{tab:model-summary}).
Distance made the largest overall contribution,
followed time and then midges.
The effect of distance manifested almost exclusively through PC1,
while the effects of midges and time where both split between PC2 and PC3.
This indicates some degree of correspondence between linear respones of the community
to midges and time, as can be seen in Figure \ref{fig:coefs}A.
In constrast, the community variation with distance
was largely uncorrelated with midges and time

The taxon-response vectors (Figure \ref{fig:pca})
provided information on the variation in overall abundance versus composition per se,
with vectors of similar direction and magnitude
implying strongly associated responses across taxa (i.e. overall abundance),
while vectors of either different direction or magnitude implying
variation in composition.
The direction of the taxon vectors was largely similar along PC2 (all positive
except for sheet weavers), reflecting the consistent overall
response of the community to midge deposition.
In contrast, the direction and magnitude
of the taxon-vectors along PC1 was quite heterogeneous,
reflecting the fact that there was large variation in composition per se with
distance from the lake.
These community-level results were a direct result of the taxon-specific responses
to the predictors (Figure \ref{fig:coefs}A),
illustrating the link between the two organizational scales.






% ---------------------------------------------------------------------------------------
% ---------------------------------------------------------------------------------------
% Discussion
% ---------------------------------------------------------------------------------------
% ---------------------------------------------------------------------------------------




\section*{Discussion}

In this study, we show how linear mixed models with temporal autocorrelation can be used
to quantify community responses to environmental variation
from replicated time series observations.
Our approach is based on previous studies that used mixed models
for estimating taxon- and community-level responses to environmental variation
through space \citep{Jackson2012, Bartrons2015}.
We extend this approach for community time series by incoporating temporal autocorrelation
with group-specific values of the autoregressive parameter,
which is important as populations with different dynamics
likely have different degrees of autocorrelation.
Furthermore, we show how predicted values from the model estimates can be combined with
with principle components analysis \citep[following][]{Jackson2012} to visualize
community composition along axes of variation most strongly associated with environmental
variation.
This analysis makes explicit the connections betweeon taxon- and community-level variation
and provides a conceptual link between the mixed model approach
and conventional ordination methods.


To illustrate its utility,
we used autoregressive mixed models to quantify the response
of a predatory arthropod community to spatiotemporal variation in allochthonous resources
at Lake M\'{y}vatn in northern Iceland.
We found a positive overall response of predator activity-density to midge deposition,
which is consistent with previous studies at M\'{y}vatn
\citep{Hoekman2011, Dreyer2012, Sanchez2018, Hoekman2019}
and expected from studies of allochthonous subsidies in other systems
\citep{}.
However, the variation among taxa in their reponses to midges was fairly limited,
which means that variation in midge deposition was primarily associated with
changes in overall abundance in the community, rather than composition per se.
In contrast, there was large variation in taxon-specific trends in activity-density
through time and space,
which manifested as spatiotemporal variation in community composition.

We found that most of the predatory arthropods showed modest levels of temporal
autocorrelation in activity density,
except for ground spiders which showed very low autocorrelation.
While we caution again strong conclusions regarding dynamic processes based on short
time series,
these results suggest that ground spiders returned more rapidly to their mean
acitivyt-density,
which in turn manifested ``faster'' flucutations through time \citep{}.
This could reflect ecological differences between grounds spiders and the other taxa,
for example in life history or in movement behavior.
Temporal autocorrelation can also be influence by species interactions,
which are not included in our model formulation.
When long time series are available,
explicit models of community dynamics can be used to infer species interactions
and so are more approach than the approach presented here.
Nonetheless, when the primary goal is to infer general responses environmental variation
(as is often the case),
our approach provides some additional insight into dynamic processes
when interpreted with caution.

[Final short paragraph to wrap things up?]
















% ---------------------------------------------------------------------------------------
% ---------------------------------------------------------------------------------------
% Acknowledgments
% ---------------------------------------------------------------------------------------
% ---------------------------------------------------------------------------------------
% You may wish to remove the Acknowledgments section while your paper
% is under review (unless you wish to waive your anonymity under
% double-blind review) if the Acknowledgments reveal your identity.
% If you remove this section, you will need to add it back in to your
% final files after acceptance.

% \section*{Acknowledgments}
%
% OEC would like to thank the world. GHC is much indebted to the solar system. AQE was supported by a generous grant from the Milky Way (MW/01010/987654).


% ---------------------------------------------------------------------------------------
% ---------------------------------------------------------------------------------------
% Appendices
% ---------------------------------------------------------------------------------------
% ---------------------------------------------------------------------------------------

% \newpage{}
%
% \input{app_A}


% ---------------------------------------------------------------------------------------
% ---------------------------------------------------------------------------------------
% Literature Cited
% ---------------------------------------------------------------------------------------
% ---------------------------------------------------------------------------------------



\bibliographystyle{ecology.bst}
\clearpage

\bibliography{refs.bib}


\clearpage

\begin{table}
\caption{\label{tab:model-summary}
Model inference.
LOO deviance was calculated from the posterior distribution of log-likelihoods
and compares the fit of the full model to a reduced model exluding either
deviations in taxon-specific responses from the mean response or
the overall response (including mean and taxon-specific components).
The scale of LOO deviance is comparable to $\Delta$AIC.
The variance partitioning was calculated using ANOVA,
with the sum-of-squares for each component scaled by the total sum-of-squares.
Paranthetical values show relative loadings for the PC axes.
The `Total' column is calculated as the sum across PC axes for each row,
with each axis weighted by the contribution of each predictor when relevant.}
\begin{tabular}{rrrrrrrr}
\toprule
 & \multicolumn{2}{c}{LOO deviance} & & \multicolumn{4}{c}{Variance partitioning} \\
 \cmidrule{2-3} \cmidrule{5-8}
 & \multicolumn{1}{c}{Taxon-variation} & \multicolumn{1}{c}{Overall} & &
    \multicolumn{1}{c}{PC1} & \multicolumn{1}{c}{PC2} & \multicolumn{1}{c}{PC3} &
    \multicolumn{1}{c}{Total} \\
& \multicolumn{1}{c}{(random)} & \multicolumn{1}{c}{(fixed + random)} & &
    \multicolumn{1}{c}{(0.34)} & \multicolumn{1}{c}{(0.17)} &
    \multicolumn{1}{c}{(0.17)} & \multicolumn{1}{c}{(0.68)} \\
\midrule
time & 37 & 34 &  & 0.00 & 0.28 & 0.84 & (0.19)\\
distance & 64 & 64 &  & 0.99 & 0.04 & 0.01 & (0.34)\\
midges & 4 & 15 &  & 0.01 & 0.67 & 0.15 & (0.14)\\
\bottomrule
\end{tabular}
\end{table}


\clearpage

\begin{figure}
\centering
\includegraphics{../analysis/output/fig1.pdf}
\caption{\label{fig:obs-data}
Community time series data for 6 predatory arthropod taxa at Lake M\'{y}vatn
(A) Standardized activity-density of arthropods through time.
Narrow gray lines are time series grouped by site and distance.
(B) Standardized activity-density of arthropods across distance from the lake.
Thick red lines are midge catch averaged by (A) year or (B) distance.
Thick blue lines are standardized activity-density averaged by taxon and
(A) year or (B) distance.
}
\end{figure}




\clearpage

\begin{figure}
\centering
\includegraphics{../analysis/output/fig2.pdf}
\caption{\label{fig:coefs}
Model estimates.
Each panel shows parameter estimates from the autoregressive mixed model,
with postior medians (points) and 68\% uncertainty intervals
(errors bars) analogous to the coverage of standard errors.
Gray vertical lines correspond to 0.
Panel A shoows taxon-specific responses to each predictor.
The shaded region correspondes with the 68\% uncertainty interval
for the mean response across taxa (i.e. fixed effects shown in panel C).
Panel B shows the autoregressive coefficient for each taxon,
while panels C and D show posterior distributions for
the fixed effects and random effect standard deviations, respectively.
}
\end{figure}



\clearpage

\begin{figure}
\centering
\includegraphics{../analysis/output/fig3.pdf}
\caption{\label{fig:pca}
Principle components analysis (PCA) of community variation.
The PCA is based on the taxon-specific responses to time, distance, and midges inferred
from the model.
Therefore, the PC axes are aligned to maximize variation associated with
respoonses to the predictor variables, similar to ordination methoods
such as "redundancy analysis".
The observed data were then projected onto these axes so that variation accounted for
by the model could be visualized in the context of the data (B-D).
The taxon vector overlays in panel A are scaled relative to vectors in B-D
for clarity of visualization.
}
\end{figure}



\end{document}
