

\section*{Results}

On average, the predatory arthropod community responded positively to midge deposition,
with ground spiders having the most positive
response, followed by ground beetles, wolf spiders, and rove beetles.
In contrast, sheet  weavers and harvestman had reponses very close to zero.
Excluding variation in taxon-specific responses to midges provided a substantially
worse fit to the data by the conventions often used for $\Delta$AIC
(on same scale as LOO deviance), where values of 2 are considered meaningful and values
greater than 10 are considered large.
However, the standard error on the LOO deviance for taxon-specific responses to midges was
large, suggesting that evidence for improved model fit when including variation among
taxa was somewhat equivocal.
In contrast, the LOO deviance provided fairly strong evidence for an overall
community response to midges.

In contrast to the response to midges, the average response across taxa through
time and space was indistinguishable from zero while the variation among taxa was
quite large.
This was especially true for distance, with rove beetles being most abundant near
the lake while wolf and ground spiders were most abundant far from the lake.
Note that the inferred variation with distance controlled for responses to midge
deposition, which declined with distance from the lake.
The inferred increase in ground spiders with distance
would likely have been weaker in the absence of accounting for the response to midges,
and visa versa.
The LOO deviance for taxon-specific responses (i.e. random effects) to time and
distance was quite strong.
Furthermore, those deviances were nearly identical to those for the overall community
responses to time and distance (respectively), which indicates that most of the
improvement in model fit from including those predictors was due to taxon-specific
variation.

The first three PC axes (which explained virtually all of the variation in the
predicted values), collectively explained ~66\% of the observed community variation.
Therefore, over half of the variation in community abundance and composition was
associated with linear responses to midge deposition, time, and distance.
Variation in PC1 was primarily due to distance, while PC2 was primarily due midges.
This can be seen in figure X, where points shaded by midge deposition (panel X)
largely separate along the vertical axis, while points shaded by distance (panel X)
separate along the horizontal.
Time was most stronlgy associated with PC3 as so community patterns in time were
not apparent from the plot of PC2 against PC1 (panel X).
Overall, distance from the lake accounted for 28\% of the variation in community a
abundance and composition, followed by midge deposition at 20\% and time at 19\%.

The direction of the taxon vectors provides information on the variation in overall
abundance versus composition per se; the more similar the direction the more
similar the response across taxa.
The direction of the taxon vectors is largely similar along PC2 (all positive
except for sheet weavers), reflecting the fact that there was a strong overall
response of the community to midge deposition.
In contrast, direction of the taxon-vectors along PC1 was quite heterogeneous,
reflecting the fact that there was large variation in composition per se with
distance from the lake.
These community-level results are a directly inferrable from the
taxon-level responses to the predictors, as shown in figure X.
