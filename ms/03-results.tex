

\section*{Results}

On average, the predatory arthropod community responded positively to midge deposition,
with ground spiders having the largest response,
followed by ground beetles, rove beetles, wolf spiders, and harvestman.
In contrast, the response of sheet weavers was very close to zero.
According to the LOO deviance (on same scale as $\Delta$AIC),
there was substantial evidence for an overall community response to midges,
while only modest evidence for variation among taxa. This is consistent with
most of the taxon-specific reponses being close to the mean response across taxa.

In contrast to the responses to midge deposition, the average linear trends through
time and space were close to zero while the variation among taxa was quite large.
This was especially true for distance, with rove beetles being most abundant near
the lake while wolf and ground spiders were most abundant far from the lake.
Note that the inferred variation with distance controlled for responses to midge
deposition, which declined with distance from the lake.
The inferred increase in ground spiders (for example) with distance
would likely have been weaker in the absence of accounting for the response to midges,
and visa versa.
The LOO deviances indicated strong support for taxon-specific variation
(i.e. random effects) in responses to time and distance.
Furthermore, those deviances were very close to those for the overall community
responses to time and distance (respectively), which indicates that most of the
improvement in model fit from including those predictors was due to taxon-specific
variation.

The AR coefficients (for yearly time steps) were generally simliar across taxa
and of modest magnitude, indicating moderate levels of temoral autocorrelation.
Ground spiders were an exception,
with an AR coefficient near zero indicating limited autocorrelation across years.
The relatively low autocorrleation is apparent from the time series,
where ground spiders appear more tightly constrianed to their mean abundance
than was the case for other taxa.

The first three PC axes (which explained virtually all of the variation in the
predicted values), collectively accounted for ~68\% of the observed community variation.
Therefore, over half of the variation in community abundance and composition was
associated with linear responses to midge deposition, time, and distance.
Variation in PC1 was almost entirely due to distance, while PC2 was primarily due midges,
although with a substantial contribution from time.
This can be seen in figure X, where points shaded by midge deposition (panel X)
largely separate along the vertical axis, while points shaded by distance (panel X)
separate along the horizontal.
Time was most stronlgy associated with PC3, such that community patterns in time were
not apparent from the plot of PC2 against PC1 (panel X).
Overall, linear association with distance from the lake accounted for
34\% of the variation in community a abundance and composition,
followed by asscotions with time at 19\% and midge deposition at 14\%.

The taxon-response vectors provides information on the variation in overall
abundance versus composition per se, with vectors of similar direction and magnitude
implying strongly associated responses across taxa (i.e. overall abundance),
while vectors of either different direction or magnitude implying
variation in composition.
The direction of the taxon vectors was largely similar along PC2 (all positive
except for sheet weavers), reflecting the fact the overall
response of the community to midge deposition.
In contrast, the direction of the taxon-vectors along PC1 was quite heterogeneous,
reflecting the fact that there was large variation in composition per se with
distance from the lake.
These community-level results were a direct result of the taxon-specific responses
to the predictors (figure X),
illustrating the link between the two organizational scales.
