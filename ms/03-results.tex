

\section*{Results}

There was a positive overall response of predator activity-density to midge deposition,
as indicated by both the estimated mean response
(Figure \ref{fig:coefs}A,C)
and the modestly large LOO deviance (Table \ref{tab:model-summary}).
The variation among taxa in the reponses to midges was meaningfuly different from zero
(Figure \ref{fig:coefs}D)
but contributed relatively little to the
model fit (Table \ref{tab:model-summary}).
In contrast to the response to midges,
there was large variation among taxa in trends through time and space,
and this taxon-variation dominated the contribution to the model fit.
Some taxa with strongly positive responses to midges (e.g. ground spiders)
increased with distance from the lake,
despite the fact that midge deposition declined with distance.
This underscores the value of including multiple predictors in a single model
so that their respective effects can be disentangled.

The AR coefficients were generally simliar across taxa and of modest magnitude,
indicating moderate levels of temporal autocorrelation across years.
Ground spiders were an exception,
with an AR coefficient bording zero indicating limited autocorrelation.
The relatively low autocorrleation is apparent from the time series,
where ground spiders appear more tightly constrianed to their mean abundance
than was the case for other taxa (Figure \ref{fig:obs-data}).
Note that while the low AR coefficent for ground spiders was associated
with by far the most positive trend with time,
the \textit{magnitude} of this trend was not much larger than inferred for some other taxa,
suggesting that it was not a statistical artifact of limited identifiability.

The first three PC axes contained all of the variation
due to linear effects of the three predictors and
collectively accounted for over half of the observed community variation
in activity-density (Table \ref{tab:model-summary}).
Distance made the largest overall contribution,
followed time and then midges.
The effect of distance manifested almost exclusively through PC1,
while the effects of midges and time where both split between PC2 and PC3.
This indicates some degree of correspondence between linear respones of the community
to midges and time, as can be seen in Figure \ref{fig:coefs}A.
In constrast, the community variation with distance
was largely uncorrelated with midges and time

The taxon-response vectors (Figure \ref{fig:pca})
provided information on the variation in overall abundance versus composition per se,
with vectors of similar direction and magnitude
implying strongly associated responses across taxa (i.e. overall abundance),
while vectors of either different direction or magnitude implying
variation in composition.
The direction of the taxon vectors was largely similar along PC2 (all positive
except for sheet weavers), reflecting the consistent overall
response of the community to midge deposition.
In contrast, the direction and magnitude
of the taxon-vectors along PC1 was quite heterogeneous,
reflecting the fact that there was large variation in composition per se with
distance from the lake.
These community-level results were a direct result of the taxon-specific responses
to the predictors (Figure \ref{fig:coefs}A),
illustrating the link between the two organizational scales.
