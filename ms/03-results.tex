

\section*{Results}

There was a positive overall response of predator activity-density to midge deposition,
as indicated by both the coefficient estimates (Figure \ref{fig:coefs}A,C)
and the modestly large improvement of the full model
relative to the reduced model (LOO deviance in Table \ref{tab:model-summary}).
The response to midges was somewhat variable across taxa,
as judged by the taxon-specific coefficients (Figure \ref{fig:coefs}A)
and the posterior distribution
for the random effect standard deviation (Figure \ref{fig:coefs}D).
However, this variation among taxa made a relatively small contribution
to the model fit (Table \ref{tab:model-summary}).
In contrast to the response to midges,
there was large variation among taxa in trends through time and space,
and this taxon-specific variation dominated the contribution to the model fit.
Overall, the magnitudes of the taxon-specific responses to time and distance
were larger than the respones to midges.

The AR coefficients were generally simliar across taxa and of modest magnitude,
indicating moderate levels of temporal autocorrelation across years
(Figure \ref{fig:coefs}B).
Ground spiders were an exception,
with an AR coefficient bording zero indicating limited autocorrelation.
The relatively low autocorrleation is apparent from the time series,
where ground spiders appear more tightly constrianed to their mean abundance
than was the case for other taxa (Figure \ref{fig:obs-data}),
which manifests as less autocorrelated (i.e. ``faster'') flucutations through time
\citep{Ziebarth2010}.
Note that while the low AR coefficent for ground spiders was associated
with by far the most positive trend with time,
the \textit{magnitude} of this trend was not much larger than inferred for some other taxa,
suggesting that it was not a statistical artifact of limited identifiability.

The first three PC axes contained all of the variation
due to linear effects of the three predictors and
collectively accounted for $\sim$68\% of the observed community variation
in activity-density (Table \ref{tab:model-summary}).
Distance made the largest overall contribution,
followed time and midges.
The effect of distance manifested almost exclusively through PC1,
while the effects of midges and time where both split between PC2 and PC3.
This indicates some degree of correspondence between linear respones of the community
to midges and time, as can be seen in taxon-specific coefficients
(Figure \ref{fig:coefs}A).
In constrast, the community reponse to distance
was largely uncorrelated with the responses to time and midges.

The taxon-response vectors (Figure \ref{fig:pca})
provided information on the variation in overall abundance versus composition per se;
vectors of similar direction and magnitude
imply consistent responses across taxa (i.e. overall abundance),
while vectors of either different direction or magnitude imply variation in composition.
The direction of the taxon vectors was largely similar along PC2 (all positive
except for sheet weavers), reflecting the consistent overall
response of the community to midge deposition.
In contrast, the direction and magnitude
of the taxon-vectors along PC1 was quite heterogeneous,
reflecting the fact that there was large variation in composition with
distance from the lake.
This contrast between effects on overall abundance versus composition
resulted directly from the relative magnitudes
of fixed versus taxon-specific random effects (Figure \ref{fig:coefs}),
illustrating the link between the two organizational scales.
