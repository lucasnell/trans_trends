

\section*{Methods}

We defined a vector $\mathbf{y}$ consiting of the transformed abundances for each taxon
in each plot through time.
Because population counts are multiplicative, we took the log-transform of the abundances,
adding one to all values to accomodate zeros.
We then z-scored (subtracted mean and divided by standard deviation) the transformed
abundances within taxa,
to give the responses of different taxa on comparable scales (\cite{Jackson2012}).
The rows of $\mathbf{y}$ were grouped as individual time series
(i.e. taxon-plot combinations),
with observations within each group ordered according to time.
From this vector,
we constructed a lagged vector $\mathbf{y\sp{\prime}}$,
where the first element for each time series was zero
followed by first through the
penultimate observtions for that time series (\cite{Ives2006}).
This allowed us to specify an autoregressive model for the $i$th observation of
$\mathbf{y}$ as

\begin{equation} \label{eq:y-i}
\begin{split}
    y_i &= \mathbf{x}_i {\boldsymbol\beta}_i +
        \left[ \phi_{\text{taxon}[i]} \right]^{\text{time}_i - \text{time}\sp{\prime}_i}
        y\sp{\prime}_i + \epsilon_i \\
    \epsilon_i &\sim \mathcal{N} \left(0, \; \sigma_{\text{residual}} \right)
    \text{,}
\end{split}
\end{equation}

\noindent where $\mathbf{x}_i$ is a row vector of predictor values for the $i$th
observation, ${\boldsymbol\beta}_i$ is a column vector of coefficients,
$\phi_{\text{taxon}[i]}$ is the taxon-specific autoregressive (AR) parameter,
$\text{time}_i$ is the current time, $\text{time}\sp{\prime}_i$ is the lagged time
(defined as for  $\mathbf{y\sp{\prime}}$), and $\epsilon_i$  is the Gaussian residual
error with standard deviation $\sigma_{\text{residual}}$.
Because $y_i$ consists of log-transformed abundances, equation \ref{eq:y-i} is
essentially a log-linear model of population dynamics, with intrinsic growth rate
$\mathbf{x}_i{\boldsymbol\beta}_i$ and density-dependence $\phi_{\text{taxon}[i]}$
(values closer to zero indicate stronger regulation).
The exponentiation of $\phi_{\text{taxon}[i]}$ by the change in time allows the model
to accomodate unequal time steps.
The primary difference between equation \ref{eq:y-i} and a more traditional AR model
is the use of separate values of the AR paramter for different groups (in this case,
different taxa).
This is an important extension, as it allows different members of a community with
different dynamics (and likely different levels of autocorrelation) to be included
in a single model.
We parameterized the model with the predictors giving the change in $y_i$,
rather than the mean of the stationary distribution as is conventionally done
(\cite{Harvey1990, Ives2006}),
because this reduced the correlation between the estimates of temporal trends
(contained by ${\boldsymbol\beta}_i$; see below) and AR parameters.

We were interested in characterizing the response of the predatory arthropod
community to midge depositon, while also accounting for potential trends through
time and space due to other factors.
Therefore, we defined the vector of predictor values for the $i$th observation
of $\mathbf{y}$ as

\begin{equation} \label{eq:x-vec}
    \mathbf{x}_i = \begin{bmatrix}
        1 & \text{midge}_i & \text{time}_i & \text{dist}_i
    \end{bmatrix}\text{,}
\end{equation}

\noindent with $1$'s corresponding to intercepts,
$\text{midge}_i$ and $\text{dist}_i$ log-transformed, and all three predictors
z-scored.
We defined a vector of corresponding coefficients defined as

\begin{equation} \label{eq:beta-vec}
{\boldsymbol\beta}_i = \begin{bmatrix}
    {}^1\hspace*{-1pt}\beta_i \\
    {}^\text{midge}\hspace*{-1pt}\beta_i \\
    {}^\text{time}\hspace*{-1pt}\beta_i \\
    {}^\text{dist}\hspace*{-1pt}\beta_i
    \end{bmatrix}\text{.}
\end{equation}

\noindent We modeled the coefficients hierarchically (following \cite{Jackson2012}) as

\begin{equation} \label{eq:betas}
\begin{split}
    {}^1\hspace*{-2pt}\beta_i &= {}^1\hspace*{-1pt}\alpha +
        {}^1\hspace*{-1pt}\zeta_{\text{taxon}[i]} +
        {}^1\hspace*{-1pt}\zeta_{\text{trans}[i]} +
        {}^1\hspace*{-1pt}\zeta_{\text{plot}[i]} \\
    {}^\text{midge}\hspace*{-1pt}\beta_i &= {}^\text{midge}\hspace*{-1pt}\alpha +
            {}^\text{midge}\hspace*{-1pt}\zeta_{\text{taxon}[i]} \\
    {}^\text{time}\hspace*{-1pt}\beta_i &= {}^\text{time}\hspace*{-1pt}\alpha +
            {}^\text{time}\hspace*{-1pt}\zeta_{\text{taxon}[i]} \\
    {}^\text{dist}\hspace*{-1pt}\beta_i &= {}^\text{dist}\hspace*{-1pt}\alpha +
            {}^\text{dist}\hspace*{-1pt}\zeta_{\text{taxon}[i]}
\end{split}
\end{equation}

\noindent with fixed effects ${}^k\hspace*{-1pt}\alpha$ and random effects
${}^k\hspace*{-1pt}\zeta_{\text{g}[j]}$ for predictor $k$ (e.g. midges)
and the $j$th level of grouping variable $g$ (e.g. taxon).
The random effects where modeled with Gaussian distributions with
standard error $\sigma_g$:

\begin{equation} \label{eq:zetas}
    {}^k\hspace*{-1pt}\zeta_{g[j]} \sim
        \mathcal{N}\left(0, \; {}^k\hspace*{-1pt}\sigma_g \right)
\end{equation}

Using conventional mixed model terminology, this model includes random intercepts
grouped by taxon, transect (notated ``trans"), and plot, and random slopes for midges,
time, and distance grouped by taxon.
The fixed slopes give the overall response across taxa to the predictor variables,
while the corresponding random effects give the deviation for each taxon from
this overall response (\cite{Jackson2012}).
The overall response of each taxon is the sum of the fixed and random components.
Characterizing the responses for each taxon in a single model using random effects
allows for ``partial pooling" of the taxon-specific responses towards the mean response,
which reduces the noisiness of individual estimates and ameleriates concerns of
multiple comparisons (\cite{Gelman2012}) that have sometimes been raised
when responses of many taxa simultaneously.

Variation in community composition per se arises from variation in the taxon-specific
responses, and the magnitude of this
variation for predictor $k$ is given by ${}^k\hspace*{-1pt}\sigma_\text{taxon}$.
Therefore, the model characterizes both
taxon-level and community-level responses to the predictors.
To make the community responses more explicit, we generated predicted values
$\hat{\mathbf{y}}$ due to variation in midges, time, and distance as

\begin{equation} \label{eq:preds}
    \hat{y}_r = \hat{\mathbf{x}}_r \hat{{\boldsymbol\beta}}_r
\end{equation}

\noindent where $\hat{\mathbf{x}}_r$ is the $r$th row from a data frame defined
analagously to equation \ref{eq:x-vec} but with values evenly spaced across the
observed range of predictors, and $\hat{{\boldsymbol\beta}}_r$ is a
corresponding vector of coefficients using the estimated fixed and random effects
as in equation \ref{eq:betas} but excluding the random intercepts grouped
by transect and plot. We then converted these predicted values into a
``replicate-by-taxon" matrix (analagous to a ``site-by
-species" matrix) and performed a Principle Components Analysis (PCA) to
generate orthogonal axes characterizating the expected
variation in the community due to the predictors (similar to \cite{Jackson2012}).
Finally, we projected the observed data onto these axes.
This allowed us to visualize variation in the community, both in terms of
composition per se and overall abundance, along axes of variation most
strongly associated with the predictor variables of the model
(i.e. midges, time, and distance).
This is analagous to more conventional ordination approaches (e.g. RDA),
but makes explicit the connection to taxon-specific responses.
Furthermore, it takes advantage  of the statisitcal machinery of
regression-type models, which allows for the formal accounting of temporal
autocorrelation that is important for valid statistical inference when using
time-series data (\cite{Ives2006}).
As specified above, our approach includes both changes in composition per se
(the typical focus of ordination analyses) and the overall community response.
This is a strength, as it is difficult to assess the relevance of variation in
composition per se without the context of the overall response.
However, the method can be modified to only include changes in composition
per se by excluding the fixed effects from the generation of model predictions
in equation \ref{eq:preds}, while retaining in the fixed effects in the model fitting.

We fit the models in a Bayesian framework using Stan, although this method could be
readily adapted to other model fitting software increasingly used by ecologists
such as WinBUGS, JAGS, and AD Model Builder.
We used Gaussian distributions with mean 0 and standard deviation 1 as
weakly-informative priors (\cite{Gelman2017})
for fixed effects, random effects, and standard deviations,
with the latter truncated as 0 as standard deviations must be positive.
We used a similar prior for the AR parameter, but with standard deviation 0.5 to aid with
convergence towards stationarity (i.e. $|\phi_{\text{taxon}[i]}|<1$) without
providing an explicit upper bound.
However, we did truncate the lower bound of the prior at 0 to prevent negative
values that would give quasi-cyclic behviors.
We assessed convergence using the standard diagnostics provided by Stan,
including effective sample size of the Markov chain, number of divergent
transitions, and potential scale reduction factor.

After fitting the full model to the observed data, we compared it to reduced models
to evaluate the strength of evidence for community responses to midges, time,
and distance.
We specified the reduced models by excluding coefficients associated with the
predictors in two ways: (1) excluding the taxon-specific deviations from the mean
response (i.e. random effects) and (2) excluding the overall response across the
full community (i.e. fixed and random effects).
Because the goal was to provide inference on the full model, rather than identify
an ``optimal" model, we only compared the full model to reduced models
excluding terms associated with a single predictor.
We compared models using an approxmation to Leave-One-Out (LOO) cross validation,
based on the posterior distribution of log-likelihoods for each model
(\cite{Vehtari2017}).
We report twice the difference in the ``estimated log predictive density" between
the full and reduced models (hereafter ``LOO deviance"), which is in units of
deviance and can interpreted in a manner similar to the difference in the
Akaike Information Critera between model pairs (i.e. $\Delta$AIC) as
commonly used in frequentist settings.

