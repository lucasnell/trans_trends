
\section*{Abstract}

Times-series data for ecological communities are increasingly available
from medium- and long-term studies designed to track responses
to environmental change. However, classical multivariate methods for
analyzing community composition are generally inappropriate for time series,
as they do not account for temporal autocorrelation
in the abundance of members of the community.
Furthermore, these traditional approaches often obscure the connections between
taxon- and community-level responses, limiting the capacity to
infer the mechanisms of community change.
We show how linear mixed models with group-specific temporal autocorrelation
can be be used to infer taxon- and community-level responses
to predictor variables from replicated time series data.
Taxon-specific responses to environmental predictors are modeled
using random effects, and these responses can be used to characterize
variation in community abundance and composition.
Furthermore, the degree of autocorrelation is
estimated separately for each taxon
to account for potential differences in their dynamics.
We illustrate the utility of the approach
by analyzing the response of a predatory arthropod community to
spatiotemporal variation in allochthonous resources in a tundra landscape.
Our results show how mixed models with temporal autocorrelation provide
a unified approach to characterizing taxon and community-level
responses to environmental variation through time and space.
