
\section*{Abstract}

Times series data for ecological communities are increasingly available
from medium- and long-term studies designed to track community reponses
to environmental change. However, classical multivariate and ordinations methods for
analyzing community composition are generally iappropriate for time series data,
as they do not account for temporal autocorrelation
in the abundance of members of the community.
Furthermore, these traditional approaches often obscure the connection between
taxon- and community-level responses, limiting the capcity to
infer the mechanisms of community change.
In this study, we show how linear mixed models with temporal autocorrelation
can be be used to infer taxon- and community-level responses
to predictor variables from replicated time series data.
Taxon-specific responses to environemtnal predictors are modeled
using random effects, which in turn can be used to characterize
variation in commumnity abundance and composition.
Furthermore, the degree of autocorrelation is
estimated separately for each taxon,
to account for the fact that different taxa are likely to be characterized by
different dynamics. We illustrate the utility of the approach
by analyzing the response of a predatory arthropod community to
spatiotemporal variation in allothchonous resources in a tundra landscape.
Our results show how mixed models with temporal autocorrelation provide
a unified approach to characterizing taxon and community-level
responses to environmental drivers through time and space.
