

\section*{Discussion}

In this study, we show how linear mixed models with temporal autocorrelation can be used
to quantify community responses to environmental variation
from replicated time series observations.
Our approach is based on previous studies that used mixed models
for estimating taxon- and community-level responses to environmental variation
through space \citep{Jackson2012, Bartrons2015}.
We extend this approach for community time series by incoporating temporal autocorrelation
with group-specific values of the autoregressive parameter,
which is important as populations with different dynamics
likely have different degrees of autocorrelation.
Furthermore, we show how predicted values from the model estimates can be combined with
with principle components analysis \citep[following][]{Jackson2012} to visualize
community composition along axes of variation most strongly associated with environmental
variation.
This analysis makes explicit the connections betweeon taxon- and community-level variation
and provides a conceptual link between the mixed model approach
and conventional ordination methods.


To illustrate its utility,
we used autoregressive mixed models to quantify the response
of a predatory arthropod community to spatiotemporal variation in allochthonous resources
at Lake M\'{y}vatn in northern Iceland.
We found a positive overall response of predator activity-density to midge deposition,
which is consistent with previous studies at M\'{y}vatn
\citep{Hoekman2011, Dreyer2012, Sanchez2018, Hoekman2019}
and expected from studies of allochthonous subsidies in other systems
\citep{}.
However, the variation among taxa in their reponses to midges was fairly limited,
which means that variation in midge deposition was primarily associated with
changes in overall abundance in the community, rather than composition per se.
In contrast, there was large variation in taxon-specific trends in activity-density
through time and space,
which manifested as spatiotemporal variation in community composition.

We found that most of the predatory arthropods showed modest levels of temporal
autocorrelation in activity density,
except for ground spiders which showed very low autocorrelation.
While we caution again strong conclusions regarding dynamic processes based on short
time series,
these results suggest that ground spiders returned more rapidly to their mean
acitivyt-density,
which in turn manifested "faster" flucutations through time \citep{}.
This could reflect ecological differences between grounds spiders and the other taxa,
for example in life history or in movement behavior.
Temporal autocorrelation can also be influence by species interactions,
which are not included in our model formulation.
When long time series are available,
explicit models of community dynamics can be used to infer species interactions
and so are more approach than the approach presented here.
Nonetheless, when the primary goal is to infer general responses environmental variation
(as is often the case),
our approach provides some additional insight into dynamic processes
when interpreted with caution.

[Final short paragraph to wrap things up?]








