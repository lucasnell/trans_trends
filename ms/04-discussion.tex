

\section*{Discussion}

In this study, we show how linear mixed effects models with group-specific
temporal autocorrelation can be used to quantify community responses to environmental
variation from replicate time series.
Our approach extends existing methodologies in two primary respects.
First, our formulation of linear mixed models includes variation in
the degree of temporal autocorrelation across different groupings (e.g., taxa),
which differs from conventional formulations of mixed models as implemented in
statsitical software widely used by ecologists (e.g. nlme).
Second, by applying such models to time-series observations of ecological communities,
we show how taxon- and community-level responses to environmental variation
can be estimated while apporpriately accounting for the autocorrelation
structure of the data.
Furthermore, we show how predicted values from the model estimates can be combined with
with principle components analysis \citep[following][]{Jackson2012} to visualize
community composition along axes of variation most strongly associated with environmental
variation.
This provides a conceptual link between our approach and conventional ordination methods.

To illustrate its utility,
we used this approach to quantify the response of a predatory arthropod community to
spatiotemporal variation in allochthonous resources at
Lake M\'{y}vatn in northern Iceland.
We found a positive overall response of predator activity-density to midge deposition,
which is consistent with previous studies at M\'{y}vatn ()
and expected from studies of allochthonous subsidies in other systems ().
However, the variation among taxa in their reponses to midges was fairly limited,
which means that variation in midge deposition was primarily associated with
changes in overall abundance, rather than composition per se.
In contrast, there was large variation in taxon-specific trends in activity-density
through time and space,
which manifested as spatiotemporal variation in community composition.



The application autoregressive mixed models to the M\'{y}vatn arthropod community
illustrates a couple of general issues for characterizing community responses
to environmental variation.







