

\section*{Introduction}


A central goal of ecology is to understand how communities vary through time, and
how this variation relates to environmental variables.
Typical methods for analyzing community patterns use aggregate measures (e.g., PCA, NMDS)
that reduce the dimensionality of the dataset.
These methods are of little use in understanding the taxon-specific responses
that most ecologists are likely to be interested in.
Fitting separate regression models for each taxon addresses this problem but
is difficult to use for among-taxa comparisons.
A more recent method uses a mixed-effects model to estimate both
community- and taxon-level responses to environmental variables.
However, these methods do not account for temporal autocorrelation so cannot be used
for time series data.




