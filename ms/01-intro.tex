

\section*{Introduction}


A central goal of ecology is to understand how communities vary through time, and
how this variability is shaped by the environment.
Time series data are increasingly being used to assess temporal trends in communities.
Because environmental variables also change through time, time series data provide
additionally valuable information for how communities change in response to the
environment. For example...

The response of ecological communities to environmental predictors is often analyzed using
ordination methods (e.g.), which map variation in abundance or occurnace onto orthogonal
axes that provide synoptic assessments of community variation. However, these methods
are generally inapporpriate for time series data, as they do not appropriately account
for temporal autocorrelation in the abundance of members of the community.
While some time-series methods have been developed for analyzing ecological communities,
these have generally focused on inferring interactions between species and therefore
require very long time series. In contrast, community time series are often short but
contain replication through space (or experimental replicates), with the goal of inferring
the responses of the community to some external driver or experimental manipulation.

Here, we show linear mixed models with temporal autocorrelation structures
(hereafter "autoregressive mixed  models") can be be used to infer taxon- and
community-level responses to multiple predictor variables from time series data. The
approach is based on the method of \cite{Jackson2012}, which models variation in
taxon-specific responses to predictors as random effects. These taxon-specific responses
can then be translated to the community level, thereby providing a unified statistical
framework for understanding the ecological responses to environmental drivers at multiple
organizational scales. We extend this approach by showing how linear mixed model can be
formulated with separate temporal autocorrelation structures for each taxon,
accounting for the fact that different taxa are likely to be characterized by
different dynamics.

We illustrate the utility of this approach by analyzing the responses of predatory
arthropods to spatiotemporal variation in allothchonous resources
at Lake M\`{y}vatn in northern Iceland. M\`{y}vatn has large emergences of midges
(Diptera: Chironomidae) that serve as food for the terrestrial arthropods.
The midges have large interannual fluctuations and decline in deposition
with distance from the lakeshore, and we were interested in separating the response
of the community to midge deposition from variation in time and space.
This study shows how autoregressive mixed models can be used to dissentangle
taxon- and community-level responses to multiple predictors from time-series data.






