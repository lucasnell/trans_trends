

\section*{Introduction}


A central goal of ecology is to understand how communities vary through time, and
how this variability is shaped by the environment.
Time series data are increasingly being used to assess temporal trends in communities.
Because environmental variables also change through time, time series data provide
additionally valuable information for how communities change in response to the
environment.


However, community time series present some statistical challenges.
First, these datasets often contain complicated correlation structures.
Correlations in community time series could be due to
autocorrelation between successive points in time,
interactions among taxa,
or physical/chemical relationships between environmental variables.


Communities are also high-dimensional in nature.
Typical methods for analyzing community patterns use aggregate measures (e.g., PCA, NMDS)
that reduce the dimensionality of the dataset.
Although these methods help in understanding changes in community composition,
they are of little use in understanding the taxon-specific responses
that most ecologists are likely to be interested in.
Fitting separate regression models for each taxon addresses this problem but
is difficult to use for among-taxa comparisons.
A more recent method uses a mixed-effects model to estimate both
community- and taxon-level responses to environmental variables \citep{Jackson2012}.
However, these methods do not account for temporal autocorrelation.


% Probably should explain more about the model here.
Here, we show how an autoregressive mixed model describes both community- and
taxon-level responses to predictors.
% =======================================================================================
We analyze a dataset
\ldots
<JAMIE>
\ldots
% =======================================================================================







