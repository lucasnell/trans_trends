\section*{Introduction}

A central goal of ecology is to understand how communities vary through time, and
how this variability is shaped by the environment.
Times-series data for ecological communities are increasingly available
from medium- and long-term studies designed to track community responses
to environmental change. For example...

The response of ecological communities to environmental predictors is often analyzed using
ordination methods (e.g., NMDS, RDA),
which map variation in abundance or occurrence onto orthogonal
axes that provide synoptic assessments of community variation \citep{Mcgarigal2013}.
However, these methods are generally inappropriate for time series data,
as they do not account
for temporal autocorrelation in the abundance of members of the community \citep{Ives2006}.
While some time-series methods have been developed for analyzing ecological communities,
these have generally focused on inferring interactions between species and therefore
require very long time series \citep{Ives1999, Hampton2013}.
In contrast, community time series are often short but
contain replication through space or across experimental units, with the goal of inferring
the responses of communities to external drivers or experimental manipulations.

Here, we show how linear mixed models with temporal autocorrelation structures
(hereafter ``autoregressive mixed models'') can be be used to infer taxon- and
community-level responses to predictor variables from replicated time series data. The
approach is based on the method of \cite{Jackson2012},
which models variation in taxon-specific responses to predictors as
random effects.
We extend this approach for time series data by formulating the model with separate
temporal autocorrelation structures for different groupings (i.e., taxa),
accounting for the fact that different taxa are likely to be characterized by
different dynamics. We then show how the model can be used to visualize
community variation in a manner analagous to ordination
\citep[following ][]{Jackson2012}.
This approach has the advantage of making the connection between
taxon- and community-level variation explicit, while accounting for the autocorrelation
structure of time series data.

We illustrate the utility of this method by analyzing the responses of predatory
arthropods to spatiotemporal variation in allothchonous resources \citep{Polis1997}
at Lake M\'{y}vatn in northern Iceland \citep{Einarsson2004}.
M\'{y}vatn has large emergences of midges
(Chironomidae) that subsidize the terrestrial plant \citep{Gratton2008}
and arthropod \citep{Dreyer2012} communities.
The midges have large interannual fluctuations \citep{Gardarsson2004}
and decline in deposition with distance from the lakeshore \citep{Dreyer2015}.
We were interested in assessing the community response to
the highly variable midge subsidy, while also accounting for linear trends across years
and distance from the lake.
This study shows how autoregressive mixed models provide a unified approach
for characterizing ecological responses to environmental drivers through time and space.
