

\section*{Introduction}


A central goal of ecology is to understand how communities vary through time, and
how this variability is shaped by the environment.
Time series data are increasingly being used to assess temporal trends in communities.
Because environmental variables also change through time, time series data provide
additionally valuable information for how communities change in response to the
environment.


However, community time series present some statistical challenges.
First, these datasets often contain complicated correlation structures.
Correlations in community time series could be due to
autocorrelation between successive points in time,
interactions among taxa,
or physical/chemical relationships between environmental variables.


Second, community data are necessarily high-dimensional, as observations for each taxon
represent a separate variable.
Typical methods for analyzing community patterns use aggregate measures (e.g., PCA, NMDS)
that reduce the dimensionality of the dataset.
Although these methods help in understanding changes in community composition,
they are of little use in understanding the taxon-specific responses
that most ecologists are likely to be interested in.
Fitting separate regression models for each taxon addresses this problem but
is difficult to use for among-taxa comparisons.
A more recent method uses a mixed-effects model to estimate both
community- and taxon-level responses to environmental variables \citep{Jackson2012}.
However, this does not account for the autocorrelation between observations through time.


Here, we extend previous mixed-effects methods to account for temporal autocorrelation.
We use an autoregressive mixed model to assess the effects of environmental variables
on both individual taxa and an entire community, as well as how they change through time.
Autoregressive parameters are estimated for each taxon, as are the coefficients
associated with each environmental variable.
We then illustrate how the model can be used to visualize community composition in
a way that is analogous to ordination.
The advantage of this approach is that it accounts for the time series structure and
makes explicit the connection between the taxa and the community as a whole.
% =======================================================================================
We analyze a dataset
\ldots
<JAMIE>
\ldots
% =======================================================================================
We use this dataset to illustrate the additional insights that can be gained
by using an autoregressive mixed model to analyze community time series.







