\documentclass[12pt]{article}
% \usepackage[sc]{mathpazo} % Like Palatino with extensive math support
\usepackage[letterpaper, margin=1in]{geometry}
% \usepackage{mathptmx} % Like Times New Roman
\usepackage{newtxtext,newtxmath}
\usepackage{fullpage}
\usepackage[authoryear,sectionbib,sort]{natbib}



\usepackage[utf8]{inputenc}
\usepackage{lineno}
\usepackage{titlesec}
\titleformat{\section}[block]{\Large\bfseries\filcenter}{\thesection}{1em}{}
\titleformat{\subsection}[block]{\Large\itshape\filcenter}{\thesubsection}{1em}{}
\titleformat{\subsubsection}[block]{\large\itshape}{\thesubsubsection}{1em}{}
\titleformat{\paragraph}[runin]{\itshape}{\theparagraph}{1em}{}[. ]\renewcommand{\refname}{Literature Cited}


% Figures
\usepackage{graphicx}




\usepackage{amsmath} % for split math environment


% Everything starts with S
\renewcommand{\thefigure}{S\arabic{figure}}
\renewcommand{\theequation}{S\arabic{equation}}
\renewcommand{\thetable}{S\arabic{table}}
\setcounter{equation}{0}
\setcounter{figure}{0}
\setcounter{table}{0}


\begin{document}

\raggedright

\textbf{Supporting Information.}
Phillips, J.S., L.A. Nell, and J.C. Botsch.
Quantifying community responses to environmental variation from replicate
time series.
\emph{Ecology}.

\vspace{12pt}

\textbf{Appendix S1: Supplementary results}

\vspace{24pt}


\begin{figure}[h!]
\centering
\includegraphics[width=0.75\textwidth]{../analysis/output/figS1.pdf}
\caption{\label{fig:pca-supp}
Principal components analysis (PCA) of community variation, showing
model predictors and observed data projected onto the PC axes.
Panel rows separate the three pairwise combinations of PC axes, and
columns separate the predictors.
Points indicate observed data, and color indicates
each observation's predictor value.
The PCA is based on the taxon-specific responses to time, distance, and midges,
as inferred from the model.
For clarity of visualization, the vector overlays are scaled relative to
the observed data projections.
}
\end{figure}




\end{document}
