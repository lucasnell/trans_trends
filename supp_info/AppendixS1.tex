\documentclass[12pt]{article}
% \usepackage[sc]{mathpazo} % Like Palatino with extensive math support
\usepackage[letterpaper, margin=1in]{geometry}
% \usepackage{mathptmx} % Like Times New Roman
\usepackage{newtxtext,newtxmath}
\usepackage{fullpage}
\usepackage[authoryear,sectionbib,sort]{natbib}



\usepackage[utf8]{inputenc}
\usepackage{lineno}
\usepackage{titlesec}
\titleformat{\section}[block]{\Large\bfseries\filcenter}{\thesection}{1em}{}
\titleformat{\subsection}[block]{\Large\itshape\filcenter}{\thesubsection}{1em}{}
\titleformat{\subsubsection}[block]{\large\itshape}{\thesubsubsection}{1em}{}
\titleformat{\paragraph}[runin]{\itshape}{\theparagraph}{1em}{}[. ]\renewcommand{\refname}{Literature Cited}


% Figures
\usepackage{graphicx}




\usepackage{amsmath} % for split math environment


% Everything starts with S
\renewcommand{\thefigure}{S\arabic{figure}}
\renewcommand{\theequation}{S\arabic{equation}}
\renewcommand{\thetable}{S\arabic{table}}
\setcounter{equation}{0}
\setcounter{figure}{0}
\setcounter{table}{0}


\begin{document}

\raggedright

\textbf{Supporting Information.}
Phillips, J.S., L.A. Nell, and J.C. Botsch.
Quantifying community responses to environmental variation from replicate
time series.
\emph{Ecology}.

\vspace{12pt}

\textbf{Appendix S1: Supplementary results}

\vspace{24pt}


\begin{figure}[h!]
\centering
\includegraphics{../analysis/output/figS1.pdf}
\caption{\label{fig:pca-13}
Principal components analysis (PCA) of community variation, projected onto
PC1 and PC3, showing (a) taxon-response vectors and
(b--d) model predictors projected onto the PC axes.
The PCA is based on the taxon-specific responses to time, distance, and midges,
as inferred from the model.
Therefore, the PC axes are aligned to maximize variation associated with responses
to the predictor variables, similar to ordination methods such as ``redundancy analysis.''
The observed data were then projected onto these axes so that variation accounted for
by the model could be visualized in the context of the data.
The taxon vector overlays in panel a are scaled relative to vectors in
b--d for clarity of visualization.
}
\end{figure}



\end{document}
